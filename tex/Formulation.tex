% !TEX root = ../Individual_Project.tex
% Formulation


% Governing equation for mass and internal energy
%dm/dt du/dt



%%%%%%%%
%%%%%%%%
%%%%%%%%

\subsection{Heat transfer across cylinder}

The main mode of heat transfer that occurs in the cylinder and that will be used throughout this report is heat conduction through the wall of the cylinder. This is modeled using:
\begin{equation}
\frac{\partial T}{\partial t} = \alpha \frac{\partial^2 T }{\partial x^2}
\end{equation}
where $a$ is the thermal diffusivity, which is defined by :
\begin{equation}
\alpha = \frac{k}{\rho c}
\end{equation}
where $k$ is the material's thermal conductivity, $\rho$ is the density of the material, and $c$ is the specific heat capacity of the material. This differential equation will be solved using forward Euler time integration as described in Section \ref{sec:numerical_methods}.

\subsection{Gas flow into cylinder}

The nozzle and corresponding gas flow will be modeled using isentropic relations, and then a discharge coefficient relationship will be used to find the approximate real values. Indeed, the isentropic Reynold's number is first calculated as follows:

\begin{equation}
\text{Re}_{\text{ideal}} = \frac{\rho_{\text{exit}}\;d_{\text{inlet}}\;u_{\text{exit}}}{\mu_{\text{exit}}}
\end{equation}
where $\rho_{\text{exit}}$, $d_{\text{inlet}}$, $u_{\text{exit}}$, and $\mu_{\text{exit}}$ are determined using real gas models as described in Section \ref{sec:property_models}. In order to find the real mass flow an empirical discharge coefficient is employed:

\begin{equation}
\text{CD} = \frac{\dot{m}_{\text{in}}}{\dot{m}_{\text{ideal}}} = A + B \:\text{Re}_{\text{ideal}} 
\end{equation}
A discharge coefficient must be used to account for the formation of a boundary layer inside the inlet tube. The empirical model that was used was obtained from \todo{Citation}, and uses the following values:

\begin{equation}
A =  0.938 ,  \quad B = -2.71
\end{equation}


\subsection{Hysteresis}

The initial work this project builds upon uses the flow at the nozzle to determine the heat transfer at the wall of the cylinder. This assumes that the hydrogen flows instantaneously from the nozzle to the wall, when in reality there of course is a time delay, measured to be approximately 2 seconds for short cylinders\todo{define short and long cylinders at some point}. This assumption is acceptable for stable inflows, but for more complex filling patterns and also for improved accuracy it becomes necessary to incorporate hysteresis. Indeed, the heat transfer coefficient at the wall in fact can be said to depend on the nozzle flow seconds prior, or more generally, on the history of the nozzle flow. This can be modeled as follows:

\begin{equation}
\frac{d}{dt}\Big(\text{Nu}\Big) = \frac{\text{Nu}_{ss}-\text{Nu}}{\tau}
\end{equation}


\subsection{Throttling}

Throttling is introduced in order to more accurately represent real life conditions. When a maximum temperature is reached, the inflow of hydrogen must be stopped in order to protect the materials of the tank, as detailed in \cref{sec:materialConstraints}. A simple method of throttling, with the flow stopping at the designated maximum temperature, 85 \degree C, and the flow restarting at a chosen temperature, in this case  75 \degree C\todo{Play around with this perhaps?}.


\subsection{Optimization}

\todo{General stuff: \\
- Check tense and person \\
- Check for repetition \\
- Check for excessive verboseness
}