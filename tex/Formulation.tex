% !TEX root = ../Individual_Project.tex
% Formulation


% Governing equation for mass and internal energy
%dm/dt du/dt

\subsection{Governing equations}

The governing equation for mass and internal energy of the gas in the tank is given by:

\begin{equation}
\frac{dU_{gas}}{dt} = \frac{d\left(m_{gas}u_{gas}\right)}{dt} = h_{in}\dot m_{in} - \dot Q_{out}
\end{equation}

\todo{consistent time derivatives}

\noindent where $U_{gas}$ is the total internal energy of the gas, $m_{gas}$ is the mass of the gas, $u_{gas}$ is the specific internal energy of the gas, $H_in$ is the enthalpy of the inlet gas, $\dot m_{in}$ the mass flow into the cylinder, and $\dot Q_{out}$ is the rate of heat transfer out of the gas to the cylinder.

The change in internal energy of the gas is equal to the difference in energy entering the system in the enthalpy of gas inflow and the energy leaving the system through the walls of the cylinder. The enthalpy of the inlet can be determined from real gas models as a function of the inlet pressure and temperature. Therefore, to obtain the internal energy variation one must calculate the gas mass flow and the heat transferred to the cylinder.

We also have conservation of mass in the system, which is expressed simply by:

\begin{equation}
\frac{dm_{gas}}{dt} = \dot m_{in}
\end{equation}

\noindent as the change in mass of the gas in the cylinder is only influenced by the influx of gas during filling.

%%%%%%%%%%%%%%%%%%%%%%%%%%%%%%%%
%%%%%%%%%%%%%%%%%%%%%%%%%%%%%%%%

\subsection{Gas mass flow into cylinder}

The nozzle and corresponding gas flow will be modeled using isentropic relations, and then a discharge coefficient relationship will be used to find the approximate real values. Indeed, the isentropic Reynold's number is first calculated as follows:

\begin{equation}
\label{equ:reynoldsIdeal}
\text{Re}_{\text{ideal}} = \frac{\rho_{exit}\;d_{inlet}\;u_{exit}}{\mu_{exit}}
\end{equation}
where $d_{\text{inlet}}$ is the diameter of the inlet delivery pipe and  $\rho_{\text{exit}}$, $u_{\text{exit}}$, and $\mu_{\text{exit}}$ are determined using real gas models as described in Section \ref{sec:property_models}. More specifically:

\begin{equation}
\rho_{exit} =  f\left(P_{exit}, S_{in}\right), \quad u_{exit} = \sqrt{2\left(H_{in} - H_{static}\right)}, \quad\mu_{exit}= f\left(P_{exit}, S_{in}\right)
\end{equation}
where $P_{exit}$ is the pressure at the end of the inlet tube, $S_{in}$ is the entropy at the inlet, $H_{in}$ the stagnation enthalpy at the inlet, $H_{static}$ the static enthalpy at the end of the inlet tube. As the system is treated as isentropic, the inlet entropy can be used to calculate exit properties. We also have:

\begin{equation}
S_{in} = f\left( P_{in}, T_{in} \right), \quad H_{in} = f\left( P_{in}, T_{in} \right), \quad H_{static} = f\left( P_{exit}, S_{in} \right)
\end{equation}

\noindent where $P_{in}$ and $T_{in}$ are the inlet pressure and temperature, respectively. The exit pressure $P_{exit}$ is taken to be the pressure inside the gas tank. However, in the case that the inlet diameter is choked, these calculations would yield velocities higher than the speed of sound, which is impossible due to the nature of the inlet pipe. For this reason, if a velocity of Mach 1 or higher is achieved at any given time, an iterative process is used to find the value of $P_{exit}$ that will yield a velocity equal to the speed of sound, and the rest of properties and ultimately the Reynolds number calculated accordingly.

In order to find the real mass flow an empirical discharge coefficient is employed.

\begin{equation}
C_D = \frac{\dot{m}_{in}}{\dot{m}_{ideal}} = c_2+ c_3 \:\text{Re}_{ideal} 
\end{equation}
A discharge coefficient must be used to account for the formation of a boundary layer inside the inlet tube. The empirical model that was used was obtained from \todo{Citation}, and uses the following values:

\begin{equation}
c_2 =  0.938 ,  \quad c_3 = -2.71
\end{equation}
From the real mass flow the actual Reynold's number of the inflow can be calculated and used to find forced convection heat transfer coefficients in \cref{sec:forcedConvection,equ:nusseltReynolds} as follows: \todo{avoid forward references}

\begin{equation}
\text{Re} = \frac{4\dot m_{in}}{\pi \mu d}
\end{equation}


%%%%%%%%%%%%%%%%%%%%%%%%%%%%%%%%
%%%%%%%%%%%%%%%%%%%%%%%%%%%%%%%%


\subsection{Heat transfer from gas to cylinder}

The heat transferred from the gas to the wall of the tank is given by a simple convection relationship:

\begin{equation}
\label{equ:convection}
\dot Q = h A \left( T_{gas} - T_{wall}\right)
\end{equation}
where $A$ is the internal surface area of the cylinder, $h$ is the heat transfer coefficient, and $T_{gas}$ and $T_{wall}$ are the temperatures of the gas and the wall, respectively. The heat transfer coefficient is a result of the combination of both forced and natural convection. This is expressed as follows, as per \cite{Incropera2007} and \cite{Kakac1987}:

\begin{equation}
h = \sqrt[4]{h_f^4 + h_n^4} 
\end{equation}

\noindent where $h_f$ is the heat transfer coefficient due to forced convection and $h_n$ is the heat transfer coefficient due to natural convection. Each coefficient can be non-dimensionalised using Nusselt's numbers, expressed as:

\begin{equation}
\text{Nu}_f = \frac{h_f D}{k}, \quad \text{Nu}_n = \frac{h_n D}{k}
\end{equation}

\noindent where $D$ is the characteristic length, in this case the diameter of the cylinder, and $k$, the thermal conductivity of the fluid. The values of the Nusselt's numbers can be determined from empirical correlations, as detailed in \cref{sec:forcedConvection,sec:naturalConvection}.

\subsubsection{Forced convection}
\label{sec:forcedConvection}
As the magnitude of heat transfer is related to the flow in forced convection, the Nusselt's number can be said to be a function of the Reynolds number, taking the form:

\begin{equation}
\label{equ:nusseltReynolds}
\text{Nu}_{f,ss} = c_4 \text{Re} ^{c_5}
\end{equation}

\noindent where ${Nu}_{f,ss}$ is the steady state Nusselt's number and the empirical constants $c_4$ and $c_5$ are given by \todo{cite} as:

\begin{equation}
\label{equ:nusseltReynoldsConsts}
c_4 =   ,  \quad c_5 = 
\end{equation}

The initial work this project builds upon uses the flow at the nozzle to determine the heat transfer at the wall of the cylinder. This assumes that the hydrogen flows instantaneously from the nozzle to the wall, when in reality there is of course a time delay. This assumption is acceptable for stable inflows, hence us referring to the steady state Nusselt's number ${Nu}_{f,ss}$,  but for more complex filling patterns and also for improved accuracy it becomes necessary to incorporate hysteresis. Indeed, the heat transfer coefficient at the wall in fact can be said to depend on the nozzle flow seconds prior, or more generally, on the history of the nozzle flow. This can be modeled as follows: \todo{Does this need to be cited / proved? Eplain better so you dont have to cite it} 

\begin{equation}
\frac{d}{dt}\Big(\text{Nu}_f \Big) = \frac{\text{Nu}_{f,ss}-\text{Nu}_f}{\tau}
\end{equation}
where $\tau$ is the time scale, which reflects the amount of time it takes for the flow to recirculate. Two time scales are considered, as two different situations may be present. The flow in the cylinder may be driven by the inflow of mass from the nozzle, in which case we have a time scale of production $\tau_{prod}$. However, if the inflow halts, the flow field in the cylinder will be driven by the dissipation of the existing flow and thus the time scale is denoted $\tau_{diss}$. Therefore the overall time scale is given by the minimum of these two: \todo{explain why it's minimum}

\begin{equation}
\tau = \text{min}\left(\tau_{prod},\tau_{diss}\right)
\end{equation}

% ED: In order to model the flow we need to consider the fluid mechanics
% in general, in recirculation flows, a recirculation time characterized by some length scale
% Need to show that recirculation happens in L/D = 3.


\paragraph{Production time scale}

The main driver of flow in the cylinder, which in turn drives convective heat transfer, is the turbulent jet that develops from the end of the nozzle throughout the length of the cylinder. Experimental results can be used to derive expressions for the axial velocity of the jet. Radial profiles of  mean axial velocity can be seen in \cref{fig:radialProfile}.

\begin{figure}[H]
\begin{centering}
\caption{Radial profiles of mean axial velocity in a turbulent round jet with Re = 95,000. Adapted from \cite{pope2000} with the data from \cite{hussein1994}}
\label{fig:radialProfile}
\end{centering}
\end{figure}


\noindent By plotting the inverse of the non-dimensional speed, $u_{exit}/u(x)$ against $x/d$ we get the clearly linear relationship shown in \cref{fig:jetSpeed}.

\begin{figure}[H]
\begin{centering}
\caption{Jet speed vs distance. Adapted from \cite{pope2000} with the data from \cite{hussein1994}}
\label{fig:jetSpeed}
\end{centering}
\end{figure}

\noindent From this experimental result the following relationship is obtained: 
\begin{equation}
\label{equ:axialSpeed}
\frac{u(x)}{u_{exit}} = \frac{c_1}{\left(x-x_0\right)/d_{inlet}}
\end{equation}

\noindent where $u(x)$ is the axial velocity at a distance $x$ from the nozzle, $x_0$ is the position of the virtual origin, and $c_1$ is an empirical constant.


The production time scale is given by the spatial integral of the axial speed of the jet, which is given in \cref{equ:axialSpeed} in \cref{sec:turbulentJets}.

\begin{equation}
\begin{aligned}
\tau_{prod} &= \int_{0}^{L} \frac{1}{u(x)} dx = \frac{1}{c_1 u_{exit}} \int_0^L \left( x - x_0 \right) dx \\[2ex]
 &= \frac{\frac{L^2}{2} - x_0L}{c_1du_{exit}}
\end{aligned}
\end{equation}

\noindent As we have $x_0\ll L$ we can write:

\begin{equation}
\tau_{prod} = \frac{L^2}{2c_1du_{exit}}
\end{equation}

\noindent and defining $c_6 = \frac{1}{2c_1}$:

\begin{equation}
\tau_{prod} = c_6 \frac{L^2}{u_{exit} d}
\end{equation}

\paragraph{Dissipation time scale}
The dissipation time scale can be said to be proportional to the length of the cylinder $L$ and the recirculation velocity $u_{recirc}$: \todo{ need to explain why?}

\begin{equation}
\tau_{diss} = c_7 \frac{L}{u_{recirc}}
\end{equation}

\noindent Let us assume that instantaneous Nusselt's number depends on $u_{recirc}$ as it depends on $\text{Re}^{c_5}$ and $c_5\sim\mathcal{O}(1)$. From \cref{equ:nusseltReynolds} we know that: 

\begin{equation}
\frac{D u_{recirc}}{\nu} = c_8 \text{Nu}_f
\end{equation}

\noindent We can therefore rewrite:

\begin{equation}
\tau_{diss} = \frac{c_7}{c_8} \frac{LD}{\nu \text{Nu}_f}
\end{equation}

\noindent Combining the two constants and absorbing the $L/D$ coefficient:
\begin{equation}
\tau_{diss} = c_9 \frac{L^2}{\nu \text{Nu}_f}
\end{equation}

\noindent We can then make use of the fact that in steady state we know $\tau_{prod} = \tau_{diss}$, which leads to:

\begin{equation}
\tau_{diss} = c_9 \frac{L^2}{\nu \text{Nu}_f} = c_6 \frac{L^2}{u_{exit} d}
\end{equation}

\noindent  Substituting the definitions of Nusselt's and Reynolds numbers from \cref{equ:reynoldsIdeal,equ:nusseltReynolds}:
\begin{equation}
c_9 = c_6 \frac{\nu\text{Nu}_f}{u_{exit}d} = c_4 c_6
\end{equation}

\noindent Which leads to a final definition of $\tau_{diss}$ in terms of literature constants:

\begin{equation}
\tau_{diss} = \frac{c_4}{2c_1}  \frac{L^2}{\nu \text{Nu}_f}
\end{equation}

\noindent Therefore, in a pure dissipation scenario, where $\text{Nu}_{f,ss} \rightarrow 0$, we have:

\begin{equation}
\frac{d}{dt}\Big(\text{Nu}_f \Big) = -\text{Nu}_f^2  \frac{2c_1}{c_4}\frac{\nu}{L^2}
\end{equation}

\noindent which can be solved by simple integration, which yields:

\begin{equation}
\int - \frac{dNu_f}{Nu_f^2} = \int \frac{2c_1}{c_4} \frac{\nu}{L^2} dt
\end{equation}

\begin{equation}
Nu_f = \frac{c_4}{2c_1} \frac{L^2}{\nu} \frac{1}{t}
\end{equation}


\subsubsection{Natural convection}
\label{sec:naturalConvection}

Natural convection is caused by buoyancy driven flow, so the natural Nusselt's number can be found to be related to the Rayleigh's number (itself a product of Grashof's and Prandt's numbers) by the following relationship:

\begin{equation}
\text{Nu}_n  = c_{10} \text{Ra}^{c_{11}}
\end{equation}
where Rayleigh's number is defined as:
\begin{equation}
\text{Ra} = \left| \frac{g\beta\left(T_{wall} - T_{gas} \right) D^3}{v\alpha}\right|
\end{equation}
where $g$ is the acceleration due to gravity, $\beta$ is the coefficient of thermal expansion, $D$ is the characteristic length, in this case the cylinder diameter, $v$ is the kinematic viscosity, and $\alpha$ is the thermal diffusivity, as defined by:
\begin{equation}
\label{equ:thermalDiffusivity}
\alpha = \frac{k}{\rho c}
\end{equation}
where $k$ is the material's thermal conductivity, $\rho$ is the density of the material, and $c$ is the specific heat capacity of the material. The coefficients are given by \todo{cite}:

\begin{equation}
c_{10} =   ,  \quad c_{11} = 
\end{equation}



%%%%%%%%%%%%%%%%%%%%%%%%%%%%%%%%
%%%%%%%%%%%%%%%%%%%%%%%%%%%%%%%%


\subsection{Heat transfer across cylinder}

The main mode of heat transfer that occurs in the cylinder and that will be used throughout this report is heat conduction through the wall of the cylinder. This is modeled using:
\begin{equation}
\label{equ:heatEquation}
\frac{\partial T}{\partial t} = \alpha \frac{\partial^2 T }{\partial x^2}
\end{equation}
where $\alpha$ is the thermal diffusivity, as defined in \cref{equ:thermalDiffusivity}. The boundary condition at the inner wall can be described by equalling the heat flux into the wall from convection to the conduction through the wall:

\begin{equation}
\label{equ:innerWallBC}
k \frac{dT_w}{dx} \bigg\rvert_{x=0} = h A \left( T_{gas} - T_{wall} \right)
\end{equation}

Solving this equation will yield the temperature distribution throughout the thickness of the cylinder wall, and more specifically, the internal wall temperature that is used to calculate the heat transferred out of the gas in \cref{equ:convection}.

\subsubsection{Outer Wall}

\subsubsection{Discretisation}

In order to solve the heat equation \gls{pde} presented in \cref{equ:heatEquation} a discretisation method must be employed. The scheme employed is known as \gls{ftcs}. 

\begin{equation}
\frac{T_{i+1,j} - T_{i,j}}{\Delta t} = \alpha \frac{T_{i,j+1}-2T_{i,j} + T_{i,j-1}}{\Delta x^2}
\end{equation}

\noindent where the $i$ and $j$ subscripts represent the nodes of the discretisation in time and space, respectively, $\Delta t$ is the time step, and $\Delta x$ is the distance between grid points. This can be rearanged for $ T_{i+1,j}$ and substituting $r = \frac{\alpha \Delta t}{ \Delta x^2}$, we have:

\begin{equation}
T_{i+1,j}= T_{i,j} +  r \left(T_{i,j+1}-2T_{i,j} + T_{i,j-1} \right)
\end{equation}

We can also apply the discretisation scheme on the boundary conditions. For the inner wall boundary condition expressed in \cref{equ:innerWallBC} we can write:

\begin{equation}
T_{i+1,j}= T_{i,j} + 2r\left(T_{i,j+1} - T_{i,j} + \frac{\Delta x h \left( T_{gas} - T_{i,j}\right) }{k_{lin}}\right)
\end{equation}


\begin{equation}
T_{i+1,j}= T_{i,j} + \frac{\Delta t}{\Delta x^2} \left( \frac{k_{lam} \left(T_{i,j+1} - T_{i,j} \right) - k_{lin} \left(T_{i,j} - T_{i,j-1} \right)}{0.5 \left( C_{lin} \rho_{lin} + C_{lam} \rho_{lam} \right)} \right)
\end{equation}



%%%%%%%%%%%%%%%%%%%%%%%%%%%%%%%%
%%%%%%%%%%%%%%%%%%%%%%%%%%%%%%%%

\subsection{Throttling}

Throttling is introduced in order to more accurately represent real life conditions. When a maximum temperature is reached, the inflow of hydrogen must be stopped in order to protect the materials of the tank, as detailed in \cref{sec:challenges}. A simple method of throttling, with the flow stopping at the designated maximum temperature, 85 \degree C, and the flow restarting at a chosen temperature. The optimum temperature at which flow restarts was considered in this study, and results are presented in \cref{sec:throttlingResults}

%%%%%%%%%%%%%%%%%%%%%%%%%%%%%%%%
%%%%%%%%%%%%%%%%%%%%%%%%%%%%%%%%

\subsection{Optimization}

\todo{General stuff: \\
- Check tense and person \\
- Check for repetition \\
- Check for excessive verboseness
- define short and long cylinders at some point
}