% !TEX root = ../Individual_Project.tex
%Introduction

\subsection{Purpose of the investigation}

%Hydrogen is a very promising alternative fuel for the future, mainly due to the absence of greenhouse emissions when burning it. There are however several concerns with hydrogen fuels, such as: risk of carried compressed fuel, obtaining hydrogen itself, etc. Furthermore, one of the most important aspects is the convenience and user experience. Indeed, even if a new technology is scientifically better, consumers will prefer an inferior technology that is more convenient to them, especially if the improved technology doesn't directly affect them (as reduced emissions don't). This is known as a negative externality, as it is a cost to a third party (in this case, society as a whole) that is not accounted for in the price of the good. For this reason, it is essential for the success of hydrogen as a fuel for its use to be as - if not more - convenient than traditional fuel.
Hydrogen is a very promising alternative fuel for the future, mainly due to the absence of greenhouse emissions when burning it. In this regard, it is superior to current petroleum-, and more generally, carbon-based fueling systems. However, air pollution is a negative externality associated with carbon-based fuels, as it is a cost to a third party (in this case, society as a whole) that is not accounted for in the price of the good. By definition, the negative consequences of polluting the air are not accounted for in the price of carbon vehicles, and therefore will have little influence on a consumer's choice. Thus, if hydrogen is to succeed as an alternative to carbon fuels, it is essential that its use be as - if not more - convenient than traditional fuel.

 One of the main aspects that currently lags behind traditional fuel is the refueling experience. Given that refueling hydrogen involves its compression, there is a significant rise in temperature, which must be kept below certain standards (358�K as per SAE J2601). This in turn leads to long refueling times, potentially lasting more than five minutes, which is cumbersome for users. Therefore, it is of prime importance to research and develop systems that enable the faster refueling of hydrogen cylinders. To this end, this project will build upon a model of filling a hydrogen cylinder which has already been developed by members of the department in order to analyse novel methods of improving fill times.

\subsection{Outline of the investigation}
	One of the current solutions to improve fill times involves cooling the hydrogen before filling the cylinder as to keep it below he maximum temperature. However, this is quite expensive, both in energy terms and in economic terms. Consequently, the aim of this project is to continue exploring several of the options available to reduce fill times and simultaneously reduce the energy consumption of the process, thus improving both convenience for users and energy efficiency of the fueling stations. Indeed, by building upon the existing cylinder model several options shall be considered, namely: refrigeration, flow regulation, heat sink usage, active cooling, heat pipe usage, and phase change materials. \todo{Update with actual options considered at end of project} From here, several options are open to further deepen or potentially broaden the investigation. An attempt to further simplify the model can be made, perhaps even reducing it so a simple algebraic relationship. Also, the model would benefit from FEA validation to aid our understanding of the heat transfer in the structure. This would go hand in hand with analyzing the temperature of the structure, and see how close this matches the gas temperature. Indeed, if the structure is at a much lower temperature than the gas, the case can be made that the current regulations are slightly erroneous, as they are meant to protect the materials of the structure, but instead regulate the gas temperature.





