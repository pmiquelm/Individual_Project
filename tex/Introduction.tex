% !TEX root = ../Individual_Project.tex
%Introduction

\subsection{Purpose of the investigation}

%Hydrogen is a very promising alternative fuel for the future, mainly due to the absence of greenhouse emissions when burning it. There are however several concerns with hydrogen fuels, such as: risk of carried compressed fuel, obtaining hydrogen itself, etc. Furthermore, one of the most important aspects is the convenience and user experience. Indeed, even if a new technology is scientifically better, consumers will prefer an inferior technology that is more convenient to them, especially if the improved technology doesn't directly affect them (as reduced emissions don't). This is known as a negative externality, as it is a cost to a third party (in this case, society as a whole) that is not accounted for in the price of the good. For this reason, it is essential for the success of hydrogen as a fuel for its use to be as - if not more - convenient than traditional fuel.
Hydrogen is a very promising alternative fuel for the future, mainly due to the absence of greenhouse emissions when burning it. In this regard, it is superior to petroleum- and, more generally, carbon-based fueling systems currently used by most road vehicles. \todo{Reword} However, air pollution is a negative externality associated with this type of fuels, as it is a cost to a third party (in this case, society as a whole) that is not accounted for in the price of the good. By definition, the consequences of polluting the air are not included in the price of traditional vehicles, and therefore will have little influence on consumers' choices. Thus, if hydrogen is to succeed as an alternative to carbon-based fuels, it is essential that its use be as convenient, if not more, than traditional fuel.

 One of the main aspects where hydrogen currently lags behind traditional fuel is the refueling process. Indeed, refueling hydrogen involves its compression, and thus a significant rise in temperature, which according to standards must be kept below a certain value (358 \degree K as per SAE J2601 \todo{Cite and include ISO}). This temperature limit in turn leads to long refueling times, potentially lasting more than five minutes, which is cumbersome for users. Therefore, it is of utmost importance to research and develop systems that enable faster refueling of hydrogen tanks. To this end, this project builds upon a model of filling a hydrogen cylinder which has already been developed by members of the Faculty of Engineering and the Environment at the Univeristy of Southampton to analyse novel methods of improving fill times.

\subsection{Outline of the investigation}
One of the current solutions to improve fill times involves cooling the hydrogen before filling the cylinder. However, this is quite an expensive process, both in energetic and economic terms. Consequently, the aim of this project is to explore several of the options available to reduce fill times while simultaneously reducing the energy consumption of the process, thus improving both convenience for users and energy efficiency of the fueling stations. 

An attempt to optimise the filling profile of hydrogen cylinders will be made by building upon the existing cylinder model. The model will be developed to include transient effects by implementing a hysteresis \hl{effect}. This will allow for innovative inlet profiles to be evaluated, such as sinusoidal or square waves \todo{What the fuck are inlet profiles}. An attempt will be further made to completely determine the ideal filling profile by means of a constrained nonlinear optimization.

% This would go hand in hand with analysing the temperature of the structure, and seeing how close this matches the gas temperature. Indeed, if the structure is at a much lower temperature than the gas, the case can be made that the current regulations are slightly erroneous, as they are meant to protect the materials of the structure, but instead regulate the gas temperature. \todo{reword this crap, out of place}


% OLD STUFF

% Indeed, by building upon the existing cylinder model several options shall be considered, namely: refrigeration, flow regulation, heat sink usage, active cooling, heat pipe usage, and phase change materials. \todo{Update with actual options considered at end of project} 


% From here, several options are open to further deepen or potentially broaden the investigation. An attempt to further simplify the model can be made, perhaps even reducing it so a simple algebraic relationship. Also, the model would benefit from FEA validation to aid our understanding of the heat transfer in the structure. 





